\section{Headroom Experiment}
\label{sec:headroom}
In this section, headroom experiment results are analyzed to guide our dynamic hybrid prefetcher design. In section \ref{sec:headroomanalysis} we will talk about the potential of our design and the difference between two headroom test. In section \ref{sec:memorybandwidthissue}, one of the most influential factor, memory bandwidth, will be discussed in detail. Some other insights will be revealed in section \ref{sec:otherinsights}.

  \subsection{Headroom Result Analysis}
  \label{sec:headroomanalysis}

  \begin{figure}[ht!]
	   \centering
	   \includegraphics[width=1.0\textwidth]{images/headroom_acc_cov.png}
	   \caption{Headroom accuracy and coverage performance}
	  \label{fig:headroom_acc_cov}
  \end{figure}

  In Fig.\ref{fig:headroom_acc_cov}, the accuracy and coverage performance of \emph{ISB, BO, NHP, Static Analysis} and \emph{Brute Force Search}. Comparing \emph{NHP} with \emph{ISB, BO}, we found that the accuracy of \emph{NHP} is significantly lower than \emph{ISB}'s, while its coverage is higher than \emph{ISB}. It indicate that \emph{NHP} issued much more prefetches than \emph{ISB} or \emph{BO}. For \emph{static analysis} part, it achieves similar accuracy with \emph{ISB}'s, and similar coverage with \emph{NHP}'s. This is understandable since static analysis focus on static data files, making wise decisions is the only thing it can do. The \emph{Brute Force Search} method focuses on speedup most. Therefore, its accuracy and coverage performance are not as good as the static ones'. Because we use \emph{SNIPER} as our simulator, the accuracy and coverage keep the same even we change the bandwidth since no prefetch is abandoned.

  Though accuracy and coverage are good metrics for evaluating prefetchers. The most import performance metric is speedup. The speedup numbers of different bandwidth are shown in Fig.\ref{fig:headroom_speedup}. Interestingly, we got results of different shapes when the bandwidth is different. \emph{ISB} has higher speedup during low bandwidth while \emph{BO} has higher during high bandwidth. The reason is that high bandwidth wouldn't give much punishment to useless prefetch while low bandwidth does. For \emph{NHP}, in both configuration it enjoys higher speedup. However, the headroom of 6.4GB/s bandwidth is more than 11\%, while only 4\% in 12.8GB/s bandwidth case. Another point worth attention is that though static method enjoys higher accuracy and coverage than the brute force one, \emph{Brute Force Search} still has better speed up. Intuitively, it is because some prefetches shouldn't have been issued by \emph{Static Analysis} due to high memory pressure at that point. The detail reason will be explained in section \ref{sec:memorybandwidthissue}.

  In our later evaluation, we will choose 6.4GB/s as our bandwidth since the headroom is quite encouraging in that case.

  %higher accuracy and coverage doesn't imply better speedup.

  \begin{figure}[ht!]
	   \centering
	   \includegraphics[width=1.0\textwidth]{images/headroom_speedup.png}
	   \caption{Headroom speedup under different memory bandwidth}
	  \label{fig:headroom_speedup}
  \end{figure}

  \subsection{Memory Bandwidth Issue}
  \label{sec:memorybandwidthissue}

  \subsection{Some other insights}
  \label{sec:otherinsights}
  Besides memory bandwidth factor, we have some other insights about these headroom experiments.
  \begin{itemize}
    \item Performance is input dependent. A well built benchmark set is important
    \item The brute force experiments are not optimal, but they can still show the preference of each PC.
    \item In the headroom experiment, PCs’ decisions are made offline and not changed during execution. And we have up to 12\% speedup headroom. If PC can make dynamic decisions, we may have more speedup.
    \item Memory bandwidth pressure affects the prefetch performance a lot. It needs to be considered and measured in dynamic hybrid prefetching system.
  \end{itemize}
