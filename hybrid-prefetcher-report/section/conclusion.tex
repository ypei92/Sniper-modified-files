\section{Conclusion and Future Work}
\label{sec:conclusion}

In this project we built a hybrid prefetching system that combines irregular and regular prefetchers, using PC localization. 
We ran two groups of headroom experiments, \emph{Brute Force Seach} and \emph{Static Analysis}. The \emph{Brute Force Search} finds the optimal decision of each PC.  
The \emph{Static Analysis} built a heuristic that uses PC statistics from ISB, BO and NHP as input, to make PC decisions closest to \emph{Brute Force Search}. 
Then this offline heuristic is used as a guide for dynamic hybrid prefetcher. 
The experiment results show that the headroom is highly related to DRAM bandwidth. On 6.4GB/s bandwidth, the headroom has on average 12\% speedup over NHP and our DHP can achieve 6\% of it. \par

We are now trying to add the bandwidth estimation to control the dynamic decisions of the hybrid system. 
As we have seen in the experiment result, currently the bandwidth-aware DHP even performs slower than DHP without it, and there is still a large gap between DH and the headroom. 
We think there are still some issues in our design of bandwidth-aware DHP. One issue might be the bandwidth estimation is incorrect. Using the number of DRAM accesses to measure bandwidth like in Section \ref{sec:memorybandwidhissue} is misleading. Because when the program is stalled by bandwidth, the simulator doesn't see any DRAM accesses issued and the bandwidth would be considered as free. However in fact, the bandwidth is fully occupied. The bandwidth would be our first step in future work. \par

The second will be adding the degree control to hybrid system. Now we only consider both ISB and BO with degree one. In the future it is necessary to increase the degree of ISB (BO's degree is at most 1).\par

\paragraph{Another Possible Direction}
During our headroom experiment design, we have been thinking about how to measure the preference of each PC. Finally our conclusion is that, because there is tradeoff between accuracy and coverage, it's difficult to decide the preference of a PC only by these two metrics. \par
Let's step back to the prefetching. There will be two possible results when a prefetch is issued: it finally hits and has positive contribution to performance, or it is finally useless and has negative contribution. Positive contribution is avoiding the latency to DRAM. Negative contribution is the occupation on bandwidth and pollution on cache. We take the positive contribution as benefit and negative contribution as cost. We can simply model the problem as\par

\begin{equation}
possible\ benefit = Benefit * Accuracy + Cost * (1-Accuracy)
\end{equation}

It would be convenient to determine which prefetcher is better if we have the benefit and cost. However, this is the most important part. The benefit and cost are dynamic and highly dependent on the system configuration, system status, benchmarks, simpoints etc. If we can estimate it using some machine learning techniques, it would be very useful. And the estimation here looks like related to the cost-aware OPT project.

